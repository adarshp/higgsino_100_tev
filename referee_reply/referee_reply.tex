\documentclass[12pt]{article}
\usepackage{fullpage}
\title{Reply to referee report on \texttt{JHEP\_111P\_0819}}
\usepackage[dvipsnames]{xcolor}
\newcommand{\Shufang}[1]{{\bf\color{Maroon}  #1}}
\newcommand{\Adarsh}[1]{{\bf\color{RoyalBlue} AP: #1}}
\newcommand{\ap}[1]{\textcolor{RoyalBlue}{#1}}
\newcommand{\issue}[1]{\bigskip\noindent\textbf{\color{Maroon}Issue #1: }}
\newcommand{\reply}{\medskip\noindent\textbf{\color{RoyalBlue}Reply:}}

\author{Adarsh Pyarelal and Shufang Su}

\begin{document}

\maketitle

We would like to thank the referee for their careful reading of the paper and their 
suggestions. The issues and our replies are below.

 \issue{1}: On page 1 the authors state ``the lightest of the
MSSM particles known as the neutralinos emerges as the most attractive
candidate for the LSP.". It is not true that the lightest MSSM particles ought
to be neutralinos (there is nothing wrong with sneutrinos being the LSP), and
also I guess that one is not hunting for an attractive candidate for LSP, but
rather trying to probe the particle nature of dark matter. I would suggest to
correct this sentence accordingly.

\reply{} What we meant was that the lightest neutralino
emerges as the most attractive candidate for the LSP. We realize that the
wording might be confusing, so we have changed it to...\Adarsh{TODO}.

 \issue{2} The authors state on page 2 that limits on
electroweakinos are around 650-750 GeV, but this is only true for a certain
mass gap among the neutralinos (e.g: 10-20 GeV such that the multi-lepton
searches are effective). From instance, the latest limits from ATLAS for pure
Higgsinos are around 150 GeV. Please correct the sentence.

\reply{} We have corrected the sentence to state that the
limits on electroweakinos are between 0.15 and 1.0 TeV, depending on the mixing
of the gauge eigenstates.

\issue{3} On page 2 the authors write The most interesting
regions of split SUSY....  This is a biased statement. Interesting because of
exactly what reason? The most interesting for who, and why? Please explain
and/or clarify.

\reply{} We acknowledge that this statement is too
strong, and have accordingly softened it, clarifying that our main interest is
in discovering thermal electroweakino dark matter.

\issue{4} On page 3 the authors make a brief overview of existing bounds on
electroweakinos at 100 TeV colliders, and a few important references are
missing. On one hand, the studies going for ”small mass splittings”
[23,25,31,32] strongly drew inspiration from the mono-jet + soft- lepton
strategies (dubbed ”soft-lepton search” by the authors) that were proposed and
studied for the LHC in e.g: 1004.4902, 1307.5952, 1310.4274, 1312.7350,
1401.1162, 1401.1235, etc. Furthermore, bounds are quoted on Winos including
disappearing track searches but not on Higgsinos (1703.05327, 1703.09675,
1901.02987 should be added), nei- ther acknowledging the chance that Higgsinos
below 100 GeV can still occur (1801.05432). This is a strange choice given that
the authors work in a Bino-Higgsino system.

\reply{} 

\issue{5} On page 5 the authors use LO kinematics normalized
to NLO via PROSPINO, but the background is only done at LO + matching. Some
processes can have large K-factors, e.g: ttbar is about 2 in the BDT, in- deed,
ttbar and tbW appear, so the significance can be overestimated. It would be
necessary to run a small sample of NLO events (which can now be done with
several MC generators in a very simple way) to check the difference in the
kinematic distributions and assess how the cut-and-count and BDT analysis are
affected. For the latter it would be enough to use the NLO events as a test
sample using the current LO training.

\reply{} 

\issue{6} On page 7 the authors discuss the background
processes. While it is clear that they have considered the relevant
”irreducible” backgrounds, they do not include any process of V+jets, with V
being a vector boson. In particular, a leptonically decaying Z plus a few jets
can mimic the final state due to missing energy generated from miss-measured
jet momenta. While the efficiency is low, the cross sections of V+jets are
fairly large, so this needs to be properly estimated.

\reply{} 

\issue{7} On page 8 the authors start a multi-variate analysis
based on boosted- decision trees (BDTs). It is not clear that the BDT would be
able to ”discover” the razor variables, e.g: Ez and Eh from Eq. 3.1 are not
explicitly included: with pT of both b’s and mbb one would still have a
degeneracy between the energy and z-momenta of the Higgs boson. So please
explain if in addition there is more information used (e.g: eta of the
b-quarks) or what motivated the choices of input variables. If the BDT includes
the razor variables, then it naively should outperform it (or have a similar
performance, in the best case).

\reply{} 

\issue{8} In the same page I find that there are very few details
about the BDT implementation. The maximal depth and minimum node size are not
written. There is no evidence of the reliability of the study, e.g: no ROC
curve is shown neither the AUC is reported. It would be desriable to present
the ROC curve, the AUC for some benchmark points, as well as illustrate the
robustness of the study by presenting e.g: the precision vs recall (or PPR vs
TPR, depending on the ML jargon used).

\reply{} 

\issue{9} The sentence about the training set is confusing, so
please clearly state if a) 75\% of the signal events and 30\% of the background
events go into training (so test is 25\% of signal and 70\% of background). If
this is what was done, what is the reason to save background events for test
purposes? Just limited MC statistics?  b) 75\% of the training events are signal
and 30\% are background (since this does not add up to 100\%, maybe a typo?) c)
please also specify the size /weights of signal and background events.
According to Table 5 in the preselection the bgd XS is about 9000 times larger
than the signal so if events are weighted according to XS then the BDT will
learn to recognize background more than focusing on signal features. One option
would be to re-train assigning the signal and background samples the same
weight.

\reply{} 

\issue{10} In page 9 the authors claim their sensitivity stems
from S/Sqrt[B], but they request 3 Monte Carlo (MC) events. This sentence is
confusing, because the number of MC events in itself is meaningless. Is it the
weight of the corresponding events what the authors mean? Please clarify it.
Moreover, to claim that 3 events is enough, this requires to have a good
control over the statistics... can the authors comment on the sample size of
both signal and background? What is the ratio of the cross section of generated
events over the one expected at FCC-hh with 3ab-1?

\reply{} 

\issue{11} In page 9 the authors display their results. If, as
stated in the introduction, dark matter is a motivation, then it iso-relic
contour lines on Fig.4 could be shown (of course, there could be a mild
dependence on other parameters like tan beta, in that case one can fix some
values). By the same token, is the model affected by current and/or future
direct and indirect detection constraints? This information is also available
in the literature, but the relevant results need to be explicitly pointed out.

\reply{} 


\end{document}
